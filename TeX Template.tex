%preamble
\documentclass{article}
\usepackage{amsmath}
\usepackage{amssymb}
\usepackage{amsthm}
\usepackage{amsfonts}
%\usepackage{times} %font; can be changed (palatino, bookman, etc.)
\usepackage{framed}
\setlength\parindent{0pt}
\usepackage{color} %for colored text, if needed
\newtheorem{thm}{Theorem}
\newtheorem{rem}{Remark}
\newtheorem{cor}{Corollary}
\newtheorem{lem}{Lemma}
\newtheorem{defn}{Definition}
\usepackage[margin=0.5in]{geometry} %margins
\usepackage[utf8]{inputenc}
\usepackage[english]{babel}
\usepackage{fancyhdr}
\usepackage{lastpage}
\usepackage{tcolorbox}
\pagestyle{fancy}

\rfoot{Page \thepage \hspace{1pt} of \pageref{LastPage}} %page numbering


\begin{document}

%title
\centerline{\huge{\bf{\LaTeX \ Template}}}
\vspace{0.2in}
\centerline{\large{Homework 1}}
\vspace{0.1in}
\centerline{Solutions written by [Name]}
\vspace{0.3in}

\begin{tcolorbox}
\noindent \textsc{Exercise 1.} Prove something.
\end{tcolorbox}


\begin{proof}
This is a proof. 
\end{proof}

\begin{rem}
The above was a proof.
\end{rem}

\begin{defn}
This is a definition.
\end{defn}

\begin{thm}
This is a theorem.
\end{thm}

\begin{proof}
This is a proof of the above theorem.

\begin{lem}
Theorems are almost always followed by a proof.
\end{lem}

\end{proof}

\begin{cor}
Proofs are fun to write. 
\end{cor}


\begin{framed}
This is framed text. You can put something important inside here. 
\end{framed}

Here is some \textcolor{red}{red text}. We can also use \textcolor{orange}{orange text}, or possibly \textcolor{blue}{blue text}. How about \textcolor{purple}{purple text}?

\vspace{0.1in} %vertical space between lines

Here is an inline equation: $\displaystyle{\sum_{i = 0}^{n} i = \frac{n(n+1)}{2}}$. Did you know I can type in \texttt{courier}?

\vspace{0.1in}

Here is a piecewise function:

$$
|x| = \begin{cases}
x & \text{if } x>0\\
0 & \text{if } x=0\\
-x& \text{if } x<0
\end{cases}
$$

Here is a matrix:

$$\mathbf{A} = \begin{pmatrix}  1 & 0 & 0 \\ 0 & 1 & 0 \\ 0 & 0 & 1 \end{pmatrix}$$

Here is a list:

\begin{itemize}
\item Here is an item.
\item Here is another item.
\end{itemize}

Here is a \emph{numbered} list:

\begin{enumerate}
\item Here is an item.
\item Here is another item.
\end{enumerate}


\end{document}